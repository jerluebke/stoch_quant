\documentclass[11pt,a4paper]{scrartcl}
\usepackage{fontspec}
\usepackage{polyglossia}
    \setdefaultlanguage{english}
\usepackage{lmodern}
\usepackage{fixcmex}
\usepackage{csquotes}
\usepackage{enumitem}
\usepackage{mathtools}
\usepackage{amssymb}
\usepackage{amsfonts}
\usepackage{textcomp}
\usepackage{gensymb}
\usepackage{siunitx}
    \sisetup{range-units=brackets}
\usepackage{physics}
\usepackage{array}
\usepackage{booktabs}
\usepackage{caption}
\usepackage{graphicx}
    \graphicspath{img}
\usepackage{tikz}
    \usetikzlibrary{calc,external}
    \tikzexternalize[prefix=extern/]
    \tikzexternaldisable
\usepackage{pgfplots}
    \pgfplotsset{%
        compat=1.16,
        table/search path={data},
    }
\usepackage[makeroom]{cancel}
\usepackage{todonotes}
\usepackage[%
    colorlinks=true, linkcolor=blue,
    % hidelinks
]{hyperref}


\setlength\parindent{0pt}


\newcommand{\tablehead}[1]{\multicolumn{1}{c}{#1}}
\newcommand*{\figref}[1]{(see fig.~\ref{#1})}
\newcommand{\eg}{e.\,g.}
\newcommand{\ie}{i.\,e.}
\newcommand{\SD}{\ensuremath{\mathcal{S}_{\mathrm{D}}}}
\newcommand{\xdoti}{\ensuremath{\dot{x}_i}}
\newcommand{\xdotj}{\ensuremath{\dot{x}_j}}
\newcommand{\xdotjm}{\ensuremath{\dot{x}_{j-1}}}
\newcommand{\deltaij}{\ensuremath{\delta_{ij}}}
\newcommand{\OverDeltaT}[1]{\ensuremath{\frac{#1}{\Delta t}}}
\newcommand{\DT}{\ensuremath{\mathrm{D}_t}}


\title{}
\subject{Some equations related to \emph{Stochastic Quantization}}
\author{Jeremiah Lübke, Franz Wilfarth}
\date{\today}


\begin{document}
\maketitle

\section*{Discrete Derivative of the Action Functional}
Let us start by discretizing the action functional:
\begin{equation*}
    \mathcal{S}\left[x(t)\right] = \int\dd{t}L\left(x(t), \dot{x}(t)\right)
    \quad\longrightarrow\quad
    \SD\left(\{x_i\}\right) = \sum_i L\left(x_i, \xdoti\right)
\end{equation*}
where $x(t) \longrightarrow x_i \equiv x(t_i)$ and $\dot{x}(t) \longrightarrow
\xdoti \equiv \OverDeltaT{x_{i}-x_{i+1}}$. For convenience we write $L_i \equiv
L(x_i, \xdoti)$.\\

Now we compute the derivative:
\begin{equation*}
    \pdv{\SD}{x_j}=\pdv{x_j}\sum_i L_i
    =\sum_i\left(\pdv{L_i}{x_i}\pdv{x_i}{x_j}+\pdv{L_i}{\xdoti}\pdv{\xdoti}{x_j}\right)
\end{equation*}
with
\begin{equation*}
    \pdv{x_i}{x_j}=\deltaij
\end{equation*}
and
\begin{equation*}
    \pdv{\xdoti}{x_j}=\pdv{\OverDeltaT{x_{i}-x_{i+1}}}{x_j}
    =\OverDeltaT{1}\left(\pdv{x_i}{x_j}-\pdv{x_{i+1}}{x_j}\right)
    =\OverDeltaT{1}\left(\deltaij-\delta_{i+1,j}\right).
\end{equation*}
The $\delta$s kill the sum, and we can write (pay attention to the indices):
\begin{align*}
    \pdv{\SD}{x_j}&=\pdv{L_j}{x_j}+\OverDeltaT{1}\left(\pdv{L_j}{\xdotj}-\pdv{L_{j-1}}{\xdotjm}\right)\\
    &=\pdv{L_j}{x_j}-\OverDeltaT{1}\left(\pdv{L_{j-1}}{\xdotjm}-\pdv{L_j}{\xdotj}\right)\\
    \implies\Aboxed{\pdv{\SD}{x_j}&=\pdv{L_j}{x_j}-\DT\left(\pdv{L_{j-1}}{\xdotjm}\right)}
\end{align*}
where $\DT(f_i)=\OverDeltaT{f_{i}-f_{i+1}}$ is the discrete time derivation
operator.\\
This is exactly what we wanted to achieve.

For clarity, we make the transition back to the continuous description:
\begin{equation*}
    \pdv{\SD}{x_j}\longrightarrow\fdv{\mathcal{S}[x]}{x}=\pdv{L}{x}-\dv{t}\pdv{L}{\dot{x}}
\end{equation*}
and receive the well known \emph{Euler-Lagrange-Equation}.

\end{document}
