\section{Results}
\subsection{Problems with the double well potential}
When attempting to find the first energy gap of the double well potential
\begin{equation}
    V(x)=h\left(x^2-a^2\right)^2
    \label{eq:double-well}
\end{equation}
where $h$ is the height of the central barrier and $2\,a$ is the distance of the
minima, one finds that it converges \emph{very} slowly (we were not able to
obtain meaningful results during our investigation).

Why is that so? At first recall, that the energy gap is computed via
approximating the correlation function
\begin{align*}
    \ev{x_0\,x_t}{0}&=\sum_{n}e^{-(E_n-E_0)\,t}\mel{0}{x_0}{n}\mel{n}{x_0}{0}
    \\&\approx{e^{-\Delta{E_1}\,t}}\abs{\mel{0}{x_0}{1}}^2
\end{align*}
which is only valid for sufficiently large succeeding energy levels, causing
the remaining terms to decay exponentially. However in our case the first two
energy levels lie rather close to each other\footnote{numerically solving the
eigenvalues of the associated Hamiltonian yields $\Delta{E_2}\approx0.79$},
invalidating the approximation.

On the other hand from a numerical point of view, the term
\begin{equation*}
    x_l(\tau_{i+1})=x_l(\tau_i)-\pdv{V(x_l)}{x_l}\Delta\tau+\dotsb
\end{equation*}
corresponds to a gradient descent, which means the algorithm samples the
minimum of a given potential to find the lowest energy gap. Now the double well
potential has two minima, which both need to be sampled sufficiently in order
to account for the splitting of the energy levels. To achieve this, the
simulated trajectory needs to tunnel back and forth between both minima.

In this context it is interesting to consider the number of transitions and
observe its change for varying heights.

\todo{desribe:
 * procedure
 * result
 * implications
}
\todo{what else?
 * modify Conclusion
 * add discrete action derivation
 * add correlation function derivation
 * add Euler-Lagrange for Euclidean Action
}
